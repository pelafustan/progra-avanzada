% Options for packages loaded elsewhere
\PassOptionsToPackage{unicode}{hyperref}
\PassOptionsToPackage{hyphens}{url}
%
\documentclass[
  spanish,
]{article}
\usepackage{lmodern}
\usepackage{amssymb,amsmath}
\usepackage{ifxetex,ifluatex}
\ifnum 0\ifxetex 1\fi\ifluatex 1\fi=0 % if pdftex
  \usepackage[T1]{fontenc}
  \usepackage[utf8]{inputenc}
  \usepackage{textcomp} % provide euro and other symbols
\else % if luatex or xetex
  \usepackage{unicode-math}
  \defaultfontfeatures{Scale=MatchLowercase}
  \defaultfontfeatures[\rmfamily]{Ligatures=TeX,Scale=1}
\fi
% Use upquote if available, for straight quotes in verbatim environments
\IfFileExists{upquote.sty}{\usepackage{upquote}}{}
\IfFileExists{microtype.sty}{% use microtype if available
  \usepackage[]{microtype}
  \UseMicrotypeSet[protrusion]{basicmath} % disable protrusion for tt fonts
}{}
\makeatletter
\@ifundefined{KOMAClassName}{% if non-KOMA class
  \IfFileExists{parskip.sty}{%
    \usepackage{parskip}
  }{% else
    \setlength{\parindent}{0pt}
    \setlength{\parskip}{6pt plus 2pt minus 1pt}}
}{% if KOMA class
  \KOMAoptions{parskip=half}}
\makeatother
\usepackage{xcolor}
\IfFileExists{xurl.sty}{\usepackage{xurl}}{} % add URL line breaks if available
\IfFileExists{bookmark.sty}{\usepackage{bookmark}}{\usepackage{hyperref}}
\hypersetup{
  pdftitle={Recuperativa Unidad II},
  pdfauthor={J. Patricio Parada G.},
  pdflang={es},
  hidelinks,
  pdfcreator={LaTeX via pandoc}}
\urlstyle{same} % disable monospaced font for URLs
\usepackage[margin=1in]{geometry}
\usepackage{color}
\usepackage{fancyvrb}
\newcommand{\VerbBar}{|}
\newcommand{\VERB}{\Verb[commandchars=\\\{\}]}
\DefineVerbatimEnvironment{Highlighting}{Verbatim}{commandchars=\\\{\}}
% Add ',fontsize=\small' for more characters per line
\usepackage{framed}
\definecolor{shadecolor}{RGB}{248,248,248}
\newenvironment{Shaded}{\begin{snugshade}}{\end{snugshade}}
\newcommand{\AlertTok}[1]{\textcolor[rgb]{0.94,0.16,0.16}{#1}}
\newcommand{\AnnotationTok}[1]{\textcolor[rgb]{0.56,0.35,0.01}{\textbf{\textit{#1}}}}
\newcommand{\AttributeTok}[1]{\textcolor[rgb]{0.77,0.63,0.00}{#1}}
\newcommand{\BaseNTok}[1]{\textcolor[rgb]{0.00,0.00,0.81}{#1}}
\newcommand{\BuiltInTok}[1]{#1}
\newcommand{\CharTok}[1]{\textcolor[rgb]{0.31,0.60,0.02}{#1}}
\newcommand{\CommentTok}[1]{\textcolor[rgb]{0.56,0.35,0.01}{\textit{#1}}}
\newcommand{\CommentVarTok}[1]{\textcolor[rgb]{0.56,0.35,0.01}{\textbf{\textit{#1}}}}
\newcommand{\ConstantTok}[1]{\textcolor[rgb]{0.00,0.00,0.00}{#1}}
\newcommand{\ControlFlowTok}[1]{\textcolor[rgb]{0.13,0.29,0.53}{\textbf{#1}}}
\newcommand{\DataTypeTok}[1]{\textcolor[rgb]{0.13,0.29,0.53}{#1}}
\newcommand{\DecValTok}[1]{\textcolor[rgb]{0.00,0.00,0.81}{#1}}
\newcommand{\DocumentationTok}[1]{\textcolor[rgb]{0.56,0.35,0.01}{\textbf{\textit{#1}}}}
\newcommand{\ErrorTok}[1]{\textcolor[rgb]{0.64,0.00,0.00}{\textbf{#1}}}
\newcommand{\ExtensionTok}[1]{#1}
\newcommand{\FloatTok}[1]{\textcolor[rgb]{0.00,0.00,0.81}{#1}}
\newcommand{\FunctionTok}[1]{\textcolor[rgb]{0.00,0.00,0.00}{#1}}
\newcommand{\ImportTok}[1]{#1}
\newcommand{\InformationTok}[1]{\textcolor[rgb]{0.56,0.35,0.01}{\textbf{\textit{#1}}}}
\newcommand{\KeywordTok}[1]{\textcolor[rgb]{0.13,0.29,0.53}{\textbf{#1}}}
\newcommand{\NormalTok}[1]{#1}
\newcommand{\OperatorTok}[1]{\textcolor[rgb]{0.81,0.36,0.00}{\textbf{#1}}}
\newcommand{\OtherTok}[1]{\textcolor[rgb]{0.56,0.35,0.01}{#1}}
\newcommand{\PreprocessorTok}[1]{\textcolor[rgb]{0.56,0.35,0.01}{\textit{#1}}}
\newcommand{\RegionMarkerTok}[1]{#1}
\newcommand{\SpecialCharTok}[1]{\textcolor[rgb]{0.00,0.00,0.00}{#1}}
\newcommand{\SpecialStringTok}[1]{\textcolor[rgb]{0.31,0.60,0.02}{#1}}
\newcommand{\StringTok}[1]{\textcolor[rgb]{0.31,0.60,0.02}{#1}}
\newcommand{\VariableTok}[1]{\textcolor[rgb]{0.00,0.00,0.00}{#1}}
\newcommand{\VerbatimStringTok}[1]{\textcolor[rgb]{0.31,0.60,0.02}{#1}}
\newcommand{\WarningTok}[1]{\textcolor[rgb]{0.56,0.35,0.01}{\textbf{\textit{#1}}}}
\usepackage{graphicx}
\makeatletter
\def\maxwidth{\ifdim\Gin@nat@width>\linewidth\linewidth\else\Gin@nat@width\fi}
\def\maxheight{\ifdim\Gin@nat@height>\textheight\textheight\else\Gin@nat@height\fi}
\makeatother
% Scale images if necessary, so that they will not overflow the page
% margins by default, and it is still possible to overwrite the defaults
% using explicit options in \includegraphics[width, height, ...]{}
\setkeys{Gin}{width=\maxwidth,height=\maxheight,keepaspectratio}
% Set default figure placement to htbp
\makeatletter
\def\fps@figure{htbp}
\makeatother
\setlength{\emergencystretch}{3em} % prevent overfull lines
\providecommand{\tightlist}{%
  \setlength{\itemsep}{0pt}\setlength{\parskip}{0pt}}
\setcounter{secnumdepth}{5}
\ifxetex
  % Load polyglossia as late as possible: uses bidi with RTL langages (e.g. Hebrew, Arabic)
  \usepackage{polyglossia}
  \setmainlanguage[]{spanish}
\else
  \usepackage[shorthands=off,main=spanish]{babel}
\fi
\ifluatex
  \usepackage{selnolig}  % disable illegal ligatures
\fi

\title{Recuperativa Unidad II}
\author{J. Patricio Parada G.}
\date{25/08/2020}

\begin{document}
\maketitle

{
\setcounter{tocdepth}{2}
\tableofcontents
}
\hypertarget{paro-carduxedaco}{%
\section{Paro cardíaco}\label{paro-carduxedaco}}

Comúnmente llamado ataque cardíaco, el paro cardíaco es una condición
riegosa y virtualmente mortífera que pone fin a millones de vidas al
año. Es una de las causas de muerte más frecuentes en humanos y se debe
a variados factores; puede ser a consecuencia del estilo de vida llevado
o debido a otras afecciones o enfermedades.

El conjunto de datos anexo presenta 12 factores que eventualmente
proporcionan información respecto a si un paciente es candidato a sufrir
un ataque cardíaco.

\hypertarget{el-dataset}{%
\section{El Dataset}\label{el-dataset}}

El conjunto de datos adjunto correponde a

\begin{Shaded}
\begin{Highlighting}[]
\NormalTok{data \textless{}{-}}\StringTok{ }\KeywordTok{read.csv}\NormalTok{(}\StringTok{"CRP\_dataset.csv"}\NormalTok{)}
\end{Highlighting}
\end{Shaded}

\hypertarget{columnas}{%
\subsection{Columnas}\label{columnas}}

Las columnas (variables) que conforman el conjunto de datos corresponden
a

\begin{Shaded}
\begin{Highlighting}[]
\KeywordTok{colnames}\NormalTok{(data)}
\end{Highlighting}
\end{Shaded}

\begin{verbatim}
##  [1] "Age"                             "Gender"                         
##  [3] "Chain_smoker"                    "Consumes_other_tobacco_products"
##  [5] "HighBP"                          "Obese"                          
##  [7] "Diabetes"                        "Metabolic_syndrome"             
##  [9] "Use_of_stimulant_drugs"          "Family_history"                 
## [11] "History_of_preeclampsia"         "CABG_history"                   
## [13] "Respiratory_illness"             "UnderRisk"
\end{verbatim}

donde:

\begin{itemize}
    \item \texttt{Age}: edad
    \item \texttt{Gender}: sexo del paciente. 1 para masculino, 2 para femenino.
    \item \texttt{Chain\_smoker}: fumador. 0 no fumador, 1 fumador.
    \item \texttt{Consumes\_other\_tobacco\_products}: consumidor de otros productos derivados del tabaco. 0 no consumidor, 1 consumidor.
    \item \texttt{HighBP}: hipertensión. 0 no hipertenso, 1 hipertenso.
    \item \texttt{Obese}: obesidad. 0 sin obesidad, 1 obeso.
    \item \texttt{Diabetes}: diabetes, 0 sin diabetes, 1 con diabetes.
    \item \texttt{Metabolic\_syndrome}: síndrome metabólico. 0 no tiene, 1 paciente con síndrome.
    \item \texttt{Use\_of\_stimulant\_drugs}: uso de drogas estimulantes. 0 no consumidor, 1 consumidor.
    \item \texttt{Family\_history}: historial familiar de paro cardíaco. 0 no tiene historial, 1 tiene historial.
    \item \texttt{History\_of\_preeclampsia}: historial de preeclampsia. 0 sin historial, 1 con historial.
    \item \texttt{CABG\_history}: historial de cirugía de bypass de arteria coronaria. 0 sin historial, 1 con historial.
    \item \texttt{Respiratory\_illness}: enfermedad respiratoria. 0 sin enferemedades respiratorias, 1 posee enfermedades respiratorias.
    \item \texttt{UnderRisk}: paciente bajo riesgo. \texttt{yes}: sí, \texttt{no}: no.
\end{itemize}

\hypertarget{estructura-del-dataset}{%
\subsection{Estructura del dataset}\label{estructura-del-dataset}}

La estructura del conjunto corresponde a

\begin{Shaded}
\begin{Highlighting}[]
\KeywordTok{str}\NormalTok{(data)}
\end{Highlighting}
\end{Shaded}

\begin{verbatim}
## 'data.frame':    889 obs. of  14 variables:
##  $ Age                            : int  48 69 53 52 48 58 42 43 41 54 ...
##  $ Gender                         : int  1 1 1 1 1 2 1 2 1 1 ...
##  $ Chain_smoker                   : int  1 0 3 0 0 0 0 0 0 0 ...
##  $ Consumes_other_tobacco_products: int  1 1 1 1 0 1 1 1 1 1 ...
##  $ HighBP                         : int  0 0 0 0 0 0 0 0 0 0 ...
##  $ Obese                          : int  1 1 1 1 0 1 0 1 1 0 ...
##  $ Diabetes                       : int  0 0 3 0 0 0 0 0 0 0 ...
##  $ Metabolic_syndrome             : int  0 0 3 0 1 0 0 0 0 0 ...
##  $ Use_of_stimulant_drugs         : int  0 0 0 0 1 0 1 0 0 1 ...
##  $ Family_history                 : int  1 1 1 1 0 1 1 1 1 1 ...
##  $ History_of_preeclampsia        : int  0 0 0 0 0 0 0 3 0 0 ...
##  $ CABG_history                   : int  0 0 0 0 0 0 3 0 0 0 ...
##  $ Respiratory_illness            : int  0 0 0 0 0 0 0 0 0 0 ...
##  $ UnderRisk                      : chr  "no" "no" "no" "no" ...
\end{verbatim}

de donde se puede observar que son 14 parámetros y 889 observaciones.

\newpage

\hypertarget{tipo-de-datos}{%
\subsection{Tipo de datos}\label{tipo-de-datos}}

De acuerdo a lo observado en el conjunto de datos y lo descrito a partir
de sus columnas, la totalidad de las variables serán consideradas como
cualitativas. Del mismo modo, se procede a filtrar los datos para saltar
las incoherencias:

\begin{Shaded}
\begin{Highlighting}[]
\NormalTok{data}\OperatorTok{$}\NormalTok{Gender[}\OperatorTok{!}\NormalTok{(data}\OperatorTok{$}\NormalTok{Gender }\OperatorTok{==}\StringTok{ }\DecValTok{1} \OperatorTok{|}\StringTok{ }\NormalTok{data}\OperatorTok{$}\NormalTok{Gender }\OperatorTok{==}\StringTok{ }\DecValTok{2}\NormalTok{)] \textless{}{-}}\StringTok{ }\OtherTok{NA}
\NormalTok{data}\OperatorTok{$}\NormalTok{Chain\_smoker[}\OperatorTok{!}\NormalTok{(data}\OperatorTok{$}\NormalTok{Chain\_smoker }\OperatorTok{==}\StringTok{ }\DecValTok{0} \OperatorTok{|}\StringTok{ }\NormalTok{data}\OperatorTok{$}\NormalTok{Chain\_smoker }\OperatorTok{==}\StringTok{ }\DecValTok{1}\NormalTok{)] \textless{}{-}}\StringTok{ }\OtherTok{NA}
\NormalTok{data}\OperatorTok{$}\NormalTok{Consumes\_other\_tobacco\_products[}
    \OperatorTok{!}\NormalTok{(data}\OperatorTok{$}\NormalTok{Consumes\_other\_tobacco\_products }\OperatorTok{==}\StringTok{ }\DecValTok{0}
    \OperatorTok{|}\StringTok{ }\NormalTok{data}\OperatorTok{$}\NormalTok{Consumes\_other\_tobacco\_products }\OperatorTok{==}\StringTok{ }\DecValTok{1}\NormalTok{)] \textless{}{-}}\StringTok{ }\OtherTok{NA}
\NormalTok{data}\OperatorTok{$}\NormalTok{HighBP[}\OperatorTok{!}\NormalTok{(data}\OperatorTok{$}\NormalTok{HighBP }\OperatorTok{==}\StringTok{ }\DecValTok{0} \OperatorTok{|}\StringTok{ }\NormalTok{data}\OperatorTok{$}\NormalTok{HighBP }\OperatorTok{==}\StringTok{ }\DecValTok{1}\NormalTok{)] \textless{}{-}}\StringTok{ }\OtherTok{NA}
\NormalTok{data}\OperatorTok{$}\NormalTok{Obese[}\OperatorTok{!}\NormalTok{(data}\OperatorTok{$}\NormalTok{Obese }\OperatorTok{==}\StringTok{ }\DecValTok{0} \OperatorTok{|}\StringTok{ }\NormalTok{data}\OperatorTok{$}\NormalTok{Obese }\OperatorTok{==}\StringTok{ }\DecValTok{1}\NormalTok{)] \textless{}{-}}\StringTok{ }\OtherTok{NA}
\NormalTok{data}\OperatorTok{$}\NormalTok{Diabetes[}\OperatorTok{!}\NormalTok{(data}\OperatorTok{$}\NormalTok{Diabetes }\OperatorTok{==}\StringTok{ }\DecValTok{0} \OperatorTok{|}\StringTok{ }\NormalTok{data}\OperatorTok{$}\NormalTok{Diabetes }\OperatorTok{==}\StringTok{ }\DecValTok{1}\NormalTok{)] \textless{}{-}}\StringTok{ }\OtherTok{NA}
\NormalTok{data}\OperatorTok{$}\NormalTok{Metabolic\_syndrome[}
    \OperatorTok{!}\NormalTok{(data}\OperatorTok{$}\NormalTok{Metabolic\_syndrome }\OperatorTok{==}\StringTok{ }\DecValTok{0}
    \OperatorTok{|}\StringTok{ }\NormalTok{data}\OperatorTok{$}\NormalTok{Metabolic\_syndrome }\OperatorTok{==}\StringTok{ }\DecValTok{1}\NormalTok{)] \textless{}{-}}\StringTok{ }\OtherTok{NA}
\NormalTok{data}\OperatorTok{$}\NormalTok{Use\_of\_stimulant\_drugs[}
    \OperatorTok{!}\NormalTok{(data}\OperatorTok{$}\NormalTok{Use\_of\_stimulant\_drugs }\OperatorTok{==}\StringTok{ }\DecValTok{0}
    \OperatorTok{|}\StringTok{ }\NormalTok{data}\OperatorTok{$}\NormalTok{Use\_of\_stimulant\_drugs }\OperatorTok{==}\StringTok{ }\DecValTok{1}\NormalTok{)] \textless{}{-}}\StringTok{ }\OtherTok{NA}
\NormalTok{data}\OperatorTok{$}\NormalTok{Family\_history[}
    \OperatorTok{!}\NormalTok{(data}\OperatorTok{$}\NormalTok{Family\_history }\OperatorTok{==}\StringTok{ }\DecValTok{0} \OperatorTok{|}\StringTok{ }\NormalTok{data}\OperatorTok{$}\NormalTok{Family\_history }\OperatorTok{==}\StringTok{ }\DecValTok{1}\NormalTok{)] \textless{}{-}}\StringTok{ }\OtherTok{NA}
\NormalTok{data}\OperatorTok{$}\NormalTok{History\_of\_preeclampsia[}
    \OperatorTok{!}\NormalTok{(data}\OperatorTok{$}\NormalTok{History\_of\_preeclampsia }\OperatorTok{==}\StringTok{ }\DecValTok{0}
    \OperatorTok{|}\StringTok{ }\NormalTok{data}\OperatorTok{$}\NormalTok{History\_of\_preeclampsia }\OperatorTok{==}\StringTok{ }\DecValTok{1}\NormalTok{)] \textless{}{-}}\StringTok{ }\OtherTok{NA}
\NormalTok{data}\OperatorTok{$}\NormalTok{CABG\_history[}\OperatorTok{!}\NormalTok{(data}\OperatorTok{$}\NormalTok{CABG\_history }\OperatorTok{==}\StringTok{ }\DecValTok{0} \OperatorTok{|}\StringTok{ }\NormalTok{data}\OperatorTok{$}\NormalTok{CABG\_history }\OperatorTok{==}\StringTok{ }\DecValTok{1}\NormalTok{)] \textless{}{-}}\StringTok{ }\OtherTok{NA}
\NormalTok{data}\OperatorTok{$}\NormalTok{Respiratory\_illness[}
    \OperatorTok{!}\NormalTok{(data}\OperatorTok{$}\NormalTok{Respiratory\_illness }\OperatorTok{==}\StringTok{ }\DecValTok{0} \OperatorTok{|}\StringTok{ }\NormalTok{data}\OperatorTok{$}\NormalTok{Respiratory\_illness }\OperatorTok{==}\StringTok{ }\DecValTok{1}\NormalTok{)] \textless{}{-}}\StringTok{ }\OtherTok{NA}
\end{Highlighting}
\end{Shaded}

Para facilitar la legibilidad de los datos, se cambia el tipo de los
parámetros a booleano.

\begin{Shaded}
\begin{Highlighting}[]
\NormalTok{data}\OperatorTok{$}\NormalTok{Age \textless{}{-}}\StringTok{ }\KeywordTok{as.factor}\NormalTok{(data}\OperatorTok{$}\NormalTok{Age)}
\NormalTok{data}\OperatorTok{$}\NormalTok{Gender \textless{}{-}}\StringTok{ }\KeywordTok{as.factor}\NormalTok{(data}\OperatorTok{$}\NormalTok{Gender)}
\NormalTok{data}\OperatorTok{$}\NormalTok{Chain\_smoker \textless{}{-}}\StringTok{ }\KeywordTok{as.logical}\NormalTok{(data}\OperatorTok{$}\NormalTok{Chain\_smoker)}
\NormalTok{data}\OperatorTok{$}\NormalTok{Consumes\_other\_tobacco\_products \textless{}{-}}\StringTok{ }\KeywordTok{as.logical}\NormalTok{(data}\OperatorTok{$}\NormalTok{Consumes\_other\_tobacco\_products)}
\NormalTok{data}\OperatorTok{$}\NormalTok{HighBP \textless{}{-}}\StringTok{ }\KeywordTok{as.logical}\NormalTok{(data}\OperatorTok{$}\NormalTok{HighBP)}
\NormalTok{data}\OperatorTok{$}\NormalTok{Obese \textless{}{-}}\StringTok{ }\KeywordTok{as.logical}\NormalTok{(data}\OperatorTok{$}\NormalTok{Obese)}
\NormalTok{data}\OperatorTok{$}\NormalTok{Diabetes \textless{}{-}}\StringTok{ }\KeywordTok{as.logical}\NormalTok{(data}\OperatorTok{$}\NormalTok{Diabetes)}
\NormalTok{data}\OperatorTok{$}\NormalTok{Metabolic\_syndrome \textless{}{-}}\StringTok{ }\KeywordTok{as.logical}\NormalTok{(data}\OperatorTok{$}\NormalTok{Metabolic\_syndrome)}
\NormalTok{data}\OperatorTok{$}\NormalTok{Use\_of\_stimulant\_drugs \textless{}{-}}\StringTok{ }\KeywordTok{as.logical}\NormalTok{(data}\OperatorTok{$}\NormalTok{Use\_of\_stimulant\_drugs)}
\NormalTok{data}\OperatorTok{$}\NormalTok{Family\_history \textless{}{-}}\StringTok{ }\KeywordTok{as.logical}\NormalTok{(data}\OperatorTok{$}\NormalTok{Family\_history)}
\NormalTok{data}\OperatorTok{$}\NormalTok{History\_of\_preeclampsia \textless{}{-}}\StringTok{ }\KeywordTok{as.logical}\NormalTok{(data}\OperatorTok{$}\NormalTok{History\_of\_preeclampsia)}
\NormalTok{data}\OperatorTok{$}\NormalTok{CABG\_history \textless{}{-}}\StringTok{ }\KeywordTok{as.logical}\NormalTok{(data}\OperatorTok{$}\NormalTok{CABG\_history)}
\NormalTok{data}\OperatorTok{$}\NormalTok{Respiratory\_illness \textless{}{-}}\StringTok{ }\KeywordTok{as.logical}\NormalTok{(data}\OperatorTok{$}\NormalTok{Respiratory\_illness)}
\end{Highlighting}
\end{Shaded}

Para el caso particular de los valores de la columna
\(\verb|UnderRisk|\), se hará uso del paquete \texttt{batman}.

\begin{Shaded}
\begin{Highlighting}[]
\ControlFlowTok{if}\NormalTok{(}\OperatorTok{!}\NormalTok{(}\StringTok{"batman"} \OperatorTok{\%in\%}\StringTok{ }\KeywordTok{rownames}\NormalTok{(}\KeywordTok{installed.packages}\NormalTok{())))\{}
    \KeywordTok{install.packages}\NormalTok{(}\StringTok{"batman"}\NormalTok{)}
\NormalTok{\}}
\KeywordTok{library}\NormalTok{(}\StringTok{"batman"}\NormalTok{)}
\end{Highlighting}
\end{Shaded}

Entonces, el filtrado de los datos de \(\verb|UnderRisk|\) se realiza
mediante

\begin{Shaded}
\begin{Highlighting}[]
\NormalTok{data}\OperatorTok{$}\NormalTok{UnderRisk \textless{}{-}}\StringTok{ }\KeywordTok{to\_logical}\NormalTok{(data}\OperatorTok{$}\NormalTok{UnderRisk)}
\end{Highlighting}
\end{Shaded}

\newpage

\hypertarget{resumen-estaduxedstico}{%
\subsection{Resumen estadístico}\label{resumen-estaduxedstico}}

El resumen estadístico después del filtrado corresponde a

\begin{Shaded}
\begin{Highlighting}[]
\KeywordTok{summary}\NormalTok{(data)}
\end{Highlighting}
\end{Shaded}

\begin{verbatim}
##       Age       Gender    Chain_smoker    Consumes_other_tobacco_products
##  50     : 40   1   :607   Mode :logical   Mode :logical                  
##  54     : 40   2   :256   FALSE:779       FALSE:144                      
##  53     : 38   NA's: 26   TRUE :107       TRUE :740                      
##  46     : 36              NA's :3         NA's :5                        
##  57     : 36                                                             
##  41     : 35                                                             
##  (Other):664                                                             
##    HighBP          Obese          Diabetes       Metabolic_syndrome
##  Mode :logical   Mode :logical   Mode :logical   Mode :logical     
##  FALSE:812       FALSE:72        FALSE:839       FALSE:850         
##  TRUE :77        TRUE :812       TRUE :49        TRUE :38          
##                  NA's :5         NA's :1         NA's :1           
##                                                                    
##                                                                    
##                                                                    
##  Use_of_stimulant_drugs Family_history  History_of_preeclampsia CABG_history   
##  Mode :logical          Mode :logical   Mode :logical           Mode :logical  
##  FALSE:812              FALSE:66        FALSE:871               FALSE:868      
##  TRUE :73               TRUE :823       TRUE :16                TRUE :19       
##  NA's :4                                NA's :2                 NA's :2        
##                                                                                
##                                                                                
##                                                                                
##  Respiratory_illness UnderRisk      
##  Mode :logical       Mode :logical  
##  FALSE:860           FALSE:698      
##  TRUE :29            TRUE :190      
##                      NA's :1        
##                                     
##                                     
## 
\end{verbatim}

\end{document}
