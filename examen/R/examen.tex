% Options for packages loaded elsewhere
\PassOptionsToPackage{unicode}{hyperref}
\PassOptionsToPackage{hyphens}{url}
%
\documentclass[
  10pt,
  spanish,
]{article}
\usepackage{lmodern}
\usepackage{amssymb,amsmath}
\usepackage{ifxetex,ifluatex}
\ifnum 0\ifxetex 1\fi\ifluatex 1\fi=0 % if pdftex
  \usepackage[T1]{fontenc}
  \usepackage[utf8]{inputenc}
  \usepackage{textcomp} % provide euro and other symbols
\else % if luatex or xetex
  \usepackage{unicode-math}
  \defaultfontfeatures{Scale=MatchLowercase}
  \defaultfontfeatures[\rmfamily]{Ligatures=TeX,Scale=1}
\fi
% Use upquote if available, for straight quotes in verbatim environments
\IfFileExists{upquote.sty}{\usepackage{upquote}}{}
\IfFileExists{microtype.sty}{% use microtype if available
  \usepackage[]{microtype}
  \UseMicrotypeSet[protrusion]{basicmath} % disable protrusion for tt fonts
}{}
\makeatletter
\@ifundefined{KOMAClassName}{% if non-KOMA class
  \IfFileExists{parskip.sty}{%
    \usepackage{parskip}
  }{% else
    \setlength{\parindent}{0pt}
    \setlength{\parskip}{6pt plus 2pt minus 1pt}}
}{% if KOMA class
  \KOMAoptions{parskip=half}}
\makeatother
\usepackage{xcolor}
\IfFileExists{xurl.sty}{\usepackage{xurl}}{} % add URL line breaks if available
\IfFileExists{bookmark.sty}{\usepackage{bookmark}}{\usepackage{hyperref}}
\hypersetup{
  pdftitle={Examen},
  pdfauthor={J. Patricio Parada G.},
  pdflang={es},
  hidelinks,
  pdfcreator={LaTeX via pandoc}}
\urlstyle{same} % disable monospaced font for URLs
\usepackage[margin=0.75in]{geometry}
\usepackage{color}
\usepackage{fancyvrb}
\newcommand{\VerbBar}{|}
\newcommand{\VERB}{\Verb[commandchars=\\\{\}]}
\DefineVerbatimEnvironment{Highlighting}{Verbatim}{commandchars=\\\{\}}
% Add ',fontsize=\small' for more characters per line
\usepackage{framed}
\definecolor{shadecolor}{RGB}{248,248,248}
\newenvironment{Shaded}{\begin{snugshade}}{\end{snugshade}}
\newcommand{\AlertTok}[1]{\textcolor[rgb]{0.94,0.16,0.16}{#1}}
\newcommand{\AnnotationTok}[1]{\textcolor[rgb]{0.56,0.35,0.01}{\textbf{\textit{#1}}}}
\newcommand{\AttributeTok}[1]{\textcolor[rgb]{0.77,0.63,0.00}{#1}}
\newcommand{\BaseNTok}[1]{\textcolor[rgb]{0.00,0.00,0.81}{#1}}
\newcommand{\BuiltInTok}[1]{#1}
\newcommand{\CharTok}[1]{\textcolor[rgb]{0.31,0.60,0.02}{#1}}
\newcommand{\CommentTok}[1]{\textcolor[rgb]{0.56,0.35,0.01}{\textit{#1}}}
\newcommand{\CommentVarTok}[1]{\textcolor[rgb]{0.56,0.35,0.01}{\textbf{\textit{#1}}}}
\newcommand{\ConstantTok}[1]{\textcolor[rgb]{0.00,0.00,0.00}{#1}}
\newcommand{\ControlFlowTok}[1]{\textcolor[rgb]{0.13,0.29,0.53}{\textbf{#1}}}
\newcommand{\DataTypeTok}[1]{\textcolor[rgb]{0.13,0.29,0.53}{#1}}
\newcommand{\DecValTok}[1]{\textcolor[rgb]{0.00,0.00,0.81}{#1}}
\newcommand{\DocumentationTok}[1]{\textcolor[rgb]{0.56,0.35,0.01}{\textbf{\textit{#1}}}}
\newcommand{\ErrorTok}[1]{\textcolor[rgb]{0.64,0.00,0.00}{\textbf{#1}}}
\newcommand{\ExtensionTok}[1]{#1}
\newcommand{\FloatTok}[1]{\textcolor[rgb]{0.00,0.00,0.81}{#1}}
\newcommand{\FunctionTok}[1]{\textcolor[rgb]{0.00,0.00,0.00}{#1}}
\newcommand{\ImportTok}[1]{#1}
\newcommand{\InformationTok}[1]{\textcolor[rgb]{0.56,0.35,0.01}{\textbf{\textit{#1}}}}
\newcommand{\KeywordTok}[1]{\textcolor[rgb]{0.13,0.29,0.53}{\textbf{#1}}}
\newcommand{\NormalTok}[1]{#1}
\newcommand{\OperatorTok}[1]{\textcolor[rgb]{0.81,0.36,0.00}{\textbf{#1}}}
\newcommand{\OtherTok}[1]{\textcolor[rgb]{0.56,0.35,0.01}{#1}}
\newcommand{\PreprocessorTok}[1]{\textcolor[rgb]{0.56,0.35,0.01}{\textit{#1}}}
\newcommand{\RegionMarkerTok}[1]{#1}
\newcommand{\SpecialCharTok}[1]{\textcolor[rgb]{0.00,0.00,0.00}{#1}}
\newcommand{\SpecialStringTok}[1]{\textcolor[rgb]{0.31,0.60,0.02}{#1}}
\newcommand{\StringTok}[1]{\textcolor[rgb]{0.31,0.60,0.02}{#1}}
\newcommand{\VariableTok}[1]{\textcolor[rgb]{0.00,0.00,0.00}{#1}}
\newcommand{\VerbatimStringTok}[1]{\textcolor[rgb]{0.31,0.60,0.02}{#1}}
\newcommand{\WarningTok}[1]{\textcolor[rgb]{0.56,0.35,0.01}{\textbf{\textit{#1}}}}
\usepackage{graphicx}
\makeatletter
\def\maxwidth{\ifdim\Gin@nat@width>\linewidth\linewidth\else\Gin@nat@width\fi}
\def\maxheight{\ifdim\Gin@nat@height>\textheight\textheight\else\Gin@nat@height\fi}
\makeatother
% Scale images if necessary, so that they will not overflow the page
% margins by default, and it is still possible to overwrite the defaults
% using explicit options in \includegraphics[width, height, ...]{}
\setkeys{Gin}{width=\maxwidth,height=\maxheight,keepaspectratio}
% Set default figure placement to htbp
\makeatletter
\def\fps@figure{htbp}
\makeatother
\setlength{\emergencystretch}{3em} % prevent overfull lines
\providecommand{\tightlist}{%
  \setlength{\itemsep}{0pt}\setlength{\parskip}{0pt}}
\setcounter{secnumdepth}{5}
\ifxetex
  % Load polyglossia as late as possible: uses bidi with RTL langages (e.g. Hebrew, Arabic)
  \usepackage{polyglossia}
  \setmainlanguage[]{spanish}
\else
  \usepackage[shorthands=off,main=spanish]{babel}
\fi
\ifluatex
  \usepackage{selnolig}  % disable illegal ligatures
\fi

\title{Examen}
\author{J. Patricio Parada G.}
\date{28/08/2020}

\begin{document}
\maketitle

{
\setcounter{tocdepth}{2}
\tableofcontents
}
\hypertarget{el-paro-carduxedaco}{%
\section{El Paro cardíaco}\label{el-paro-carduxedaco}}

Comúnmente llamado ataque cardíaco, el paro cardíaco es una condición
riegosa y virtualmente mortífera que pone fin a millones de vidas al
año. Es una de las causas de muerte más frecuentes en humanos y se debe
a variados factores; puede ser a consecuencia del estilo de vida llevado
o debido a otras afecciones o enfermedades.

El conjunto de datos anexo presenta 12 factores que eventualmente
proporcionan información respecto a si un paciente es candidato a sufrir
un ataque cardíaco.

\hypertarget{el-dataset}{%
\section{El Dataset}\label{el-dataset}}

El conjunto de datos adjunto correponde a

\begin{Shaded}
\begin{Highlighting}[]
\NormalTok{df \textless{}{-}}\StringTok{ }\KeywordTok{read.csv}\NormalTok{(}\StringTok{"CRP\_dataset\_clean.csv"}\NormalTok{, }\DataTypeTok{stringsAsFactors =} \OtherTok{FALSE}\NormalTok{)}
\end{Highlighting}
\end{Shaded}

\hypertarget{columnas}{%
\subsection{Columnas}\label{columnas}}

Las variables incluídas en el conjunto de datos corresponden a

\begin{Shaded}
\begin{Highlighting}[]
\KeywordTok{colnames}\NormalTok{(df)}
\end{Highlighting}
\end{Shaded}

\begin{verbatim}
##  [1] "Age"                             "Gender"                         
##  [3] "Chain_smoker"                    "Consumes_other_tobacco_products"
##  [5] "HighBP"                          "Obese"                          
##  [7] "Diabetes"                        "Metabolic_syndrome"             
##  [9] "Use_of_stimulant_drugs"          "Family_history"                 
## [11] "History_of_preeclampsia"         "CABG_history"                   
## [13] "Respiratory_illness"             "UnderRisk"
\end{verbatim}

en donde:

\begin{itemize}
\tightlist
\item
  \texttt{Age}: Edad.
\item
  \texttt{Gender}: género, 1 para masculino y 2 femenino.
\item
  \texttt{Chain\_smoker}: fumador, 0 para no fumador y 1 en caso
  contrario.
\item
  \texttt{Consumes\_other\_tobacco\_products}: consumo de otros
  productos derivados del tabaco. 1 para consumidor, 0 para no
  consumidor.
\item
  \texttt{HighBP}: hipertensión. 0 persona sin hipertensión, 0 indica
  afección.
\item
  \texttt{Obese}: obesidad. 0 para rangos de peso normales, 1 para
  obesidad.
\item
  \texttt{Diabetes}: diabetes. 1 para diabético, 0 para no diabético.
\item
  \texttt{Metabolic\_syndrome}: híndrome metabólico. 0 para ausencia, 1
  indica presencia.
\item
  \texttt{Use\_of\_stimulant\_drugs}: hso de drogas estimulantes. 0 para
  no consumidor, 1 para consumidor.
\item
  \texttt{Family\_history}: historial familiar de ataques cardíácos. 1
  para antecedentes, 0 en caso contrario.
\item
  \texttt{History\_of\_preeclampsia}: historial de preeclampsia. 1 casi
  afirmativo, 0 negativo.
\item
  \texttt{CABG\_history}: historial de cirugía de bypass de la arteria
  coronaria. 1 para operado, 0 en caso adverso.
\item
  \texttt{Respiratory\_illness}: enfermedades respiratorias. 0 no
  presenta, 1 presenta.
\item
  \texttt{UnderRisk}: riesgoso. \texttt{yes} para sí, \texttt{no} para
  no.
\end{itemize}

\hypertarget{estructura-de-los-datos}{%
\subsection{Estructura de los datos}\label{estructura-de-los-datos}}

Las observaciones se encuentras estructuradas como

\begin{Shaded}
\begin{Highlighting}[]
\KeywordTok{str}\NormalTok{(df)}
\end{Highlighting}
\end{Shaded}

\begin{verbatim}
## 'data.frame':    889 obs. of  14 variables:
##  $ Age                            : int  84 55 80 40 45 78 77 56 999 75 ...
##  $ Gender                         : int  1 1 1 1 1 2 1 2 1 1 ...
##  $ Chain_smoker                   : int  1 0 0 0 0 0 0 0 0 0 ...
##  $ Consumes_other_tobacco_products: int  1 1 1 1 0 1 1 1 1 1 ...
##  $ HighBP                         : int  0 0 0 0 0 0 0 0 0 0 ...
##  $ Obese                          : int  1 1 1 1 0 1 0 1 1 0 ...
##  $ Diabetes                       : int  0 0 0 0 0 0 0 0 0 0 ...
##  $ Metabolic_syndrome             : int  0 0 0 0 1 0 0 0 0 0 ...
##  $ Use_of_stimulant_drugs         : int  0 0 0 0 1 0 1 0 0 1 ...
##  $ Family_history                 : int  1 1 1 1 0 1 1 1 1 1 ...
##  $ History_of_preeclampsia        : int  0 0 0 0 0 0 0 0 0 0 ...
##  $ CABG_history                   : int  0 0 0 0 0 0 0 0 0 0 ...
##  $ Respiratory_illness            : int  0 0 0 0 0 0 0 0 0 0 ...
##  $ UnderRisk                      : chr  "no" "no" "no" "no" ...
\end{verbatim}

Se puede observar el tipo de dato de cada parámetro observado, en donde
es posible ver que casi todos son de tipo \texttt{int}, a excepción del
parámetro \texttt{UnderRisk}, el cual es de tipo \texttt{char}, lo cual
es esperable a partir de la descripción de las variables dada
anteriormente.

También se indica el número de observaciones, que corresponden a 889.

\hypertarget{tipo-de-variables}{%
\subsection{Tipo de variables}\label{tipo-de-variables}}

Para efectos prácticos, serán consideradas todas las variables como
variables cualitativas, a excepción de la variable \texttt{Age}. Durante
el desarrollo del presente texto, el tipo de dato presente en el
dataframe será ajustado para facilitar operaciones.

\hypertarget{resumen-de-datos}{%
\subsection{Resumen de datos}\label{resumen-de-datos}}

Antes de realizar cualquier tipo de limpieza de los datos, se procede a
hacer un resumen estadístico de los datos en bruto:

\begin{Shaded}
\begin{Highlighting}[]
\KeywordTok{summary}\NormalTok{(df)}
\end{Highlighting}
\end{Shaded}

\begin{verbatim}
##       Age                  Gender            Chain_smoker        
##  Min.   :  0.0000000   Min.   :1.00000000   Min.   :0.000000000  
##  1st Qu.: 51.0000000   1st Qu.:1.00000000   1st Qu.:0.000000000  
##  Median : 62.0000000   Median :1.00000000   Median :0.000000000  
##  Mean   : 67.1023622   Mean   :1.30033746   Mean   :0.120359955  
##  3rd Qu.: 74.0000000   3rd Qu.:2.00000000   3rd Qu.:0.000000000  
##  Max.   :999.0000000   Max.   :2.00000000   Max.   :1.000000000  
##  Consumes_other_tobacco_products     HighBP                 Obese            
##  Min.   :0.000000000             Min.   :0.0000000000   Min.   :0.000000000  
##  1st Qu.:1.000000000             1st Qu.:0.0000000000   1st Qu.:1.000000000  
##  Median :1.000000000             Median :0.0000000000   Median :1.000000000  
##  Mean   :0.838020247             Mean   :0.0866141732   Mean   :0.919010124  
##  3rd Qu.:1.000000000             3rd Qu.:0.0000000000   3rd Qu.:1.000000000  
##  Max.   :1.000000000             Max.   :1.0000000000   Max.   :1.000000000  
##     Diabetes            Metabolic_syndrome     Use_of_stimulant_drugs
##  Min.   :0.0000000000   Min.   :0.0000000000   Min.   :0.0000000000  
##  1st Qu.:0.0000000000   1st Qu.:0.0000000000   1st Qu.:0.0000000000  
##  Median :0.0000000000   Median :0.0000000000   Median :0.0000000000  
##  Mean   :0.0551181102   Mean   :0.0427446569   Mean   :0.0821147357  
##  3rd Qu.:0.0000000000   3rd Qu.:0.0000000000   3rd Qu.:0.0000000000  
##  Max.   :1.0000000000   Max.   :1.0000000000   Max.   :1.0000000000  
##  Family_history       History_of_preeclampsia  CABG_history         
##  Min.   :0.00000000   Min.   :0.0000000000    Min.   :0.0000000000  
##  1st Qu.:1.00000000   1st Qu.:0.0000000000    1st Qu.:0.0000000000  
##  Median :1.00000000   Median :0.0000000000    Median :0.0000000000  
##  Mean   :0.92575928   Mean   :0.0179977503    Mean   :0.0213723285  
##  3rd Qu.:1.00000000   3rd Qu.:0.0000000000    3rd Qu.:0.0000000000  
##  Max.   :1.00000000   Max.   :1.0000000000    Max.   :1.0000000000  
##  Respiratory_illness     UnderRisk        
##  Min.   :0.0000000000   Length:889        
##  1st Qu.:0.0000000000   Class :character  
##  Median :0.0000000000   Mode  :character  
##  Mean   :0.0326209224                     
##  3rd Qu.:0.0000000000                     
##  Max.   :1.0000000000
\end{verbatim}

en donde se puede apreciar que las variable están dentro de los valores
esperados. La única variable que presenta un valores atípicos sería
\texttt{Age}.

Realizando un \texttt{boxplot} a la variable

\begin{Shaded}
\begin{Highlighting}[]
\KeywordTok{boxplot}\NormalTok{(df}\OperatorTok{$}\NormalTok{Age, }\DataTypeTok{main =} \StringTok{"Edad"}\NormalTok{, }\StringTok{"ylab"}\NormalTok{ =}\StringTok{ "Age"}\NormalTok{)}
\end{Highlighting}
\end{Shaded}

\includegraphics{examen_files/figure-latex/unnamed-chunk-5-1.pdf}

en donde se observan valores extremos demasiado lejanos. Ordenando los
datos de dicha columna de manera ordenada

\begin{Shaded}
\begin{Highlighting}[]
\KeywordTok{sort}\NormalTok{(df}\OperatorTok{$}\NormalTok{Age)}
\end{Highlighting}
\end{Shaded}

\begin{verbatim}
##   [1]   0   0   0   0   0   0  40  40  40  40  40  40  40  40  40  40  40  40
##  [19]  40  40  40  40  40  40  40  41  41  41  41  41  41  41  41  41  41  41
##  [37]  41  41  41  41  41  41  41  41  41  41  41  41  41  41  42  42  42  42
##  [55]  42  42  42  42  42  42  42  42  42  42  43  43  43  43  43  43  43  43
##  [73]  43  43  43  43  43  43  43  43  43  43  43  44  44  44  44  44  44  44
##  [91]  44  44  44  44  44  44  44  44  44  44  44  44  44  44  44  44  44  44
## [109]  44  45  45  45  45  45  45  45  45  45  45  45  45  45  45  45  45  45
## [127]  45  45  45  45  45  45  46  46  46  46  46  46  46  46  46  46  46  46
## [145]  46  46  46  46  46  46  47  47  47  47  47  47  47  47  47  47  47  47
## [163]  47  47  47  47  47  48  48  48  48  48  48  48  48  48  48  48  48  48
## [181]  48  48  48  49  49  49  49  49  49  49  49  49  49  49  49  49  49  49
## [199]  49  49  49  50  50  50  50  50  50  50  50  50  50  50  50  50  50  50
## [217]  50  50  50  51  51  51  51  51  51  51  51  51  51  51  51  51  51  52
## [235]  52  52  52  52  52  52  52  52  52  52  52  52  52  52  52  52  52  52
## [253]  52  52  52  52  52  52  52  52  52  52  53  53  53  53  53  53  53  53
## [271]  53  53  53  53  53  53  53  53  53  53  53  53  53  53  53  54  54  54
## [289]  54  54  54  54  54  54  54  54  54  54  54  54  54  54  54  54  54  54
## [307]  55  55  55  55  55  55  55  55  55  55  55  55  55  55  55  55  55  55
## [325]  55  55  56  56  56  56  56  56  56  56  56  56  56  56  56  56  57  57
## [343]  57  57  57  57  57  57  57  57  57  57  57  57  57  57  57  57  57  57
## [361]  57  57  57  57  57  58  58  58  58  58  58  58  58  58  58  59  59  59
## [379]  59  59  59  59  59  59  59  59  59  59  59  59  59  59  59  59  60  60
## [397]  60  60  60  60  60  60  60  60  60  60  60  60  60  60  60  60  60  61
## [415]  61  61  61  61  61  61  61  61  61  61  61  61  61  61  61  61  61  61
## [433]  61  61  62  62  62  62  62  62  62  62  62  62  62  62  62  62  62  62
## [451]  62  62  62  62  63  63  63  63  63  63  63  63  63  63  63  63  63  63
## [469]  63  63  63  63  63  63  64  64  64  64  64  64  64  64  64  64  64  64
## [487]  64  64  64  64  64  64  64  64  64  64  64  64  64  65  65  65  65  65
## [505]  65  65  65  65  65  65  65  65  65  65  65  65  65  66  66  66  66  66
## [523]  66  66  66  66  66  66  66  66  66  66  66  66  66  67  67  67  67  67
## [541]  67  67  67  67  67  67  67  67  67  68  68  68  68  68  68  68  68  68
## [559]  68  68  68  68  68  69  69  69  69  69  69  69  69  69  69  69  69  69
## [577]  69  69  69  70  70  70  70  70  70  70  70  70  70  70  70  70  70  70
## [595]  70  70  70  70  70  70  70  71  71  71  71  71  71  71  71  71  71  71
## [613]  71  71  71  71  71  71  71  71  71  72  72  72  72  72  72  72  72  72
## [631]  72  72  72  72  72  72  72  72  72  72  73  73  73  73  73  73  73  73
## [649]  73  73  73  73  73  73  73  73  73  74  74  74  74  74  74  74  74  74
## [667]  74  74  74  74  74  74  74  74  74  74  75  75  75  75  75  75  75  75
## [685]  75  75  75  75  75  75  75  75  75  75  75  75  75  75  75  75  76  76
## [703]  76  76  76  76  76  76  76  76  76  76  76  76  76  76  77  77  77  77
## [721]  77  77  77  77  77  77  77  77  77  77  78  78  78  78  78  78  78  78
## [739]  78  79  79  79  79  79  79  79  79  79  79  79  79  79  79  79  79  79
## [757]  79  79  79  79  80  80  80  80  80  80  80  80  80  80  80  80  80  80
## [775]  80  80  80  80  80  80  80  81  81  81  81  81  81  81  81  81  81  81
## [793]  81  81  81  81  81  81  81  81  81  81  81  81  82  82  82  82  82  82
## [811]  82  82  82  82  82  82  82  82  82  82  82  82  82  83  83  83  83  83
## [829]  83  83  83  83  83  83  83  83  83  83  83  83  83  83  84  84  84  84
## [847]  84  84  84  84  84  84  84  84  84  84  84  84  84  84  84  84  84  84
## [865]  84  85  85  85  85  85  85  85  85  85  85  85  85  85  85  85  85  85
## [883]  85  85 999 999 999 999 999
\end{verbatim}

se puede ver que el valor extremo 999 no tiene sentido, en consecuencia
inválido, ya que es imposible que un humano viva dicha cantidad de años.
El otro valor extremo, 0, es técnicamente válido, ya que podría
representar la edad de neonatos menores a 1 año, no será considerado
como tal por encontrarse demasiado alejado del grueso de los datos.

\begin{Shaded}
\begin{Highlighting}[]
\NormalTok{outliers \textless{}{-}}\StringTok{ }\KeywordTok{boxplot.stats}\NormalTok{(df}\OperatorTok{$}\NormalTok{Age)}\OperatorTok{$}\NormalTok{out}
\NormalTok{df}\OperatorTok{$}\NormalTok{Age \textless{}{-}}\StringTok{ }\KeywordTok{ifelse}\NormalTok{(df}\OperatorTok{$}\NormalTok{Age }\OperatorTok{\%in\%}
\StringTok{                    }\NormalTok{outliers,}
                    \OtherTok{NA}\NormalTok{, df}\OperatorTok{$}\NormalTok{Age)}
\end{Highlighting}
\end{Shaded}

Nuevamente se presenta un diagrama de caja para \texttt{Age}

\begin{Shaded}
\begin{Highlighting}[]
\KeywordTok{boxplot}\NormalTok{(df}\OperatorTok{$}\NormalTok{Age, }\DataTypeTok{main =} \StringTok{"Edad"}\NormalTok{, }\StringTok{"ylab"}\NormalTok{ =}\StringTok{ "Age"}\NormalTok{)}
\end{Highlighting}
\end{Shaded}

\includegraphics{examen_files/figure-latex/unnamed-chunk-8-1.pdf}

además del correspondiente resumen

\begin{Shaded}
\begin{Highlighting}[]
\KeywordTok{summary}\NormalTok{(df}\OperatorTok{$}\NormalTok{Age)}
\end{Highlighting}
\end{Shaded}

\begin{verbatim}
##       Min.    1st Qu.     Median       Mean    3rd Qu.       Max.       NA's 
## 40.0000000 51.0000000 62.0000000 62.2539863 74.0000000 85.0000000         11
\end{verbatim}

En donde el rango de edades ahora va de 40 a 85 años.

\hypertarget{filtrado}{%
\subsection{Filtrado}\label{filtrado}}

Para efectos prácticos, ahora se trabajará con un conjunto reducido en
que se eliminarán los valores atípicos del parámetro \texttt{Age},
manteniendo sólo las columnas \texttt{Gender}, \texttt{Chain\_smoker},
\texttt{Obese}, \texttt{Diabetes}, \texttt{Use\_of\_stimulant\_drugs},
\texttt{Family\_history} y \texttt{UnderRisk}.

\begin{Shaded}
\begin{Highlighting}[]
\NormalTok{cols \textless{}{-}}\StringTok{ }\KeywordTok{c}\NormalTok{(}
    \StringTok{"Age"}\NormalTok{, }\StringTok{"Gender"}\NormalTok{, }\StringTok{"Chain\_smoker"}\NormalTok{, }\StringTok{"Obese"}\NormalTok{,}
    \StringTok{"Diabetes"}\NormalTok{, }\StringTok{"Use\_of\_stimulant\_drugs"}\NormalTok{,}
    \StringTok{"Family\_history"}\NormalTok{, }\StringTok{"UnderRisk"}\NormalTok{)}
\NormalTok{new\_df \textless{}{-}}\StringTok{ }\KeywordTok{na.omit}\NormalTok{(df[, cols])}
\end{Highlighting}
\end{Shaded}

También se cambiará el tipo de dato de las variables cualitativas, las
que serán transformadas a lógicas, con excepción de género, que está
como factor.

\begin{Shaded}
\begin{Highlighting}[]
\NormalTok{new\_df}\OperatorTok{$}\NormalTok{Gender \textless{}{-}}\StringTok{ }\KeywordTok{as.factor}\NormalTok{(new\_df}\OperatorTok{$}\NormalTok{Gender)}
\NormalTok{new\_df}\OperatorTok{$}\NormalTok{Chain\_smoker \textless{}{-}}\StringTok{ }\KeywordTok{as.logical}\NormalTok{(new\_df}\OperatorTok{$}\NormalTok{Chain\_smoker)}
\NormalTok{new\_df}\OperatorTok{$}\NormalTok{Obese \textless{}{-}}\StringTok{ }\KeywordTok{as.logical}\NormalTok{(new\_df}\OperatorTok{$}\NormalTok{Obese)}
\NormalTok{new\_df}\OperatorTok{$}\NormalTok{Diabetes \textless{}{-}}\StringTok{ }\KeywordTok{as.logical}\NormalTok{(new\_df}\OperatorTok{$}\NormalTok{Diabetes)}
\NormalTok{new\_df}\OperatorTok{$}\NormalTok{Use\_of\_stimulant\_drugs \textless{}{-}}\StringTok{ }\KeywordTok{as.logical}\NormalTok{(new\_df}\OperatorTok{$}\NormalTok{Use\_of\_stimulant\_drugs)}
\NormalTok{new\_df}\OperatorTok{$}\NormalTok{Family\_history \textless{}{-}}\StringTok{ }\KeywordTok{as.logical}\NormalTok{(new\_df}\OperatorTok{$}\NormalTok{Family\_history)}
\end{Highlighting}
\end{Shaded}

Para el caso de \texttt{UnderRisk}, se hará uso del paquete
\texttt{batman}.

\begin{Shaded}
\begin{Highlighting}[]
\ControlFlowTok{if}\NormalTok{ (}\OperatorTok{!}\NormalTok{(}\StringTok{"batman"} \OperatorTok{\%in\%}\StringTok{ }\KeywordTok{rownames}\NormalTok{(}\KeywordTok{installed.packages}\NormalTok{()))) \{}
    \KeywordTok{install.packages}\NormalTok{(}\StringTok{"batman"}\NormalTok{)}
\NormalTok{\}}
\KeywordTok{library}\NormalTok{(}\StringTok{"batman"}\NormalTok{)}
\NormalTok{new\_df}\OperatorTok{$}\NormalTok{UnderRisk \textless{}{-}}\StringTok{ }\KeywordTok{to\_logical}\NormalTok{(new\_df}\OperatorTok{$}\NormalTok{UnderRisk)}
\end{Highlighting}
\end{Shaded}

Exponiendo nuevamente la estructura y su resumen estadístico

\begin{Shaded}
\begin{Highlighting}[]
\KeywordTok{str}\NormalTok{(new\_df)}
\end{Highlighting}
\end{Shaded}

\begin{verbatim}
## 'data.frame':    878 obs. of  8 variables:
##  $ Age                   : int  84 55 80 40 45 78 77 56 75 47 ...
##  $ Gender                : Factor w/ 2 levels "1","2": 1 1 1 1 1 2 1 2 1 1 ...
##  $ Chain_smoker          : logi  TRUE FALSE FALSE FALSE FALSE FALSE ...
##  $ Obese                 : logi  TRUE TRUE TRUE TRUE FALSE TRUE ...
##  $ Diabetes              : logi  FALSE FALSE FALSE FALSE FALSE FALSE ...
##  $ Use_of_stimulant_drugs: logi  FALSE FALSE FALSE FALSE TRUE FALSE ...
##  $ Family_history        : logi  TRUE TRUE TRUE TRUE FALSE TRUE ...
##  $ UnderRisk             : logi  FALSE FALSE FALSE FALSE FALSE FALSE ...
##  - attr(*, "na.action")= 'omit' Named int [1:11] 9 150 311 369 378 412 512 580 594 723 ...
##   ..- attr(*, "names")= chr [1:11] "9" "150" "311" "369" ...
\end{verbatim}

\begin{Shaded}
\begin{Highlighting}[]
\KeywordTok{summary}\NormalTok{(new\_df)}
\end{Highlighting}
\end{Shaded}

\begin{verbatim}
##       Age             Gender  Chain_smoker      Obese          Diabetes      
##  Min.   :40.0000000   1:613   Mode :logical   Mode :logical   Mode :logical  
##  1st Qu.:51.0000000   2:265   FALSE:772       FALSE:70        FALSE:830      
##  Median :62.0000000           TRUE :106       TRUE :808       TRUE :48       
##  Mean   :62.2539863                                                          
##  3rd Qu.:74.0000000                                                          
##  Max.   :85.0000000                                                          
##  Use_of_stimulant_drugs Family_history  UnderRisk      
##  Mode :logical          Mode :logical   Mode :logical  
##  FALSE:807              FALSE:65        FALSE:690      
##  TRUE :71               TRUE :813       TRUE :188      
##                                                        
##                                                        
## 
\end{verbatim}

\hypertarget{visualizaciuxf3n}{%
\subsection{Visualización}\label{visualizaciuxf3n}}

Ahora se verán algunas gráficas del nuevo subconjunto de datos

\begin{Shaded}
\begin{Highlighting}[]
\KeywordTok{barplot}\NormalTok{(}
    \KeywordTok{table}\NormalTok{(new\_df}\OperatorTok{$}\NormalTok{Age),}
    \DataTypeTok{main =} \StringTok{"Edad"}\NormalTok{,}
    \DataTypeTok{xlab =} \StringTok{"Edad"}\NormalTok{,}
    \DataTypeTok{ylab =} \StringTok{"Número de pacientes"}
\NormalTok{)}
\end{Highlighting}
\end{Shaded}

\includegraphics{examen_files/figure-latex/unnamed-chunk-14-1.pdf}

\begin{Shaded}
\begin{Highlighting}[]
\KeywordTok{barplot}\NormalTok{(}
    \KeywordTok{table}\NormalTok{(new\_df}\OperatorTok{$}\NormalTok{Gender),}
    \DataTypeTok{main =} \StringTok{"Género"}\NormalTok{,}
    \DataTypeTok{names.arg =} \KeywordTok{c}\NormalTok{(}\StringTok{"Masculino"}\NormalTok{, }\StringTok{"Femenino"}\NormalTok{),}
    \DataTypeTok{ylab =} \StringTok{"Número de pacientes"}
\NormalTok{)}
\end{Highlighting}
\end{Shaded}

\includegraphics{examen_files/figure-latex/unnamed-chunk-14-2.pdf}

\begin{Shaded}
\begin{Highlighting}[]
\KeywordTok{barplot}\NormalTok{(}
    \KeywordTok{table}\NormalTok{(new\_df}\OperatorTok{$}\NormalTok{Chain\_smoker),}
    \DataTypeTok{main =} \StringTok{"Fumador"}\NormalTok{,}
    \DataTypeTok{names.arg =} \KeywordTok{c}\NormalTok{(}\StringTok{"No"}\NormalTok{, }\StringTok{"Sí"}\NormalTok{),}
    \DataTypeTok{ylab =} \StringTok{"Número de pacientes"}
\NormalTok{)}
\end{Highlighting}
\end{Shaded}

\includegraphics{examen_files/figure-latex/unnamed-chunk-14-3.pdf}

\begin{Shaded}
\begin{Highlighting}[]
\KeywordTok{barplot}\NormalTok{(}
    \KeywordTok{table}\NormalTok{(new\_df}\OperatorTok{$}\NormalTok{Obese),}
    \DataTypeTok{main =} \StringTok{"Obesidad"}\NormalTok{,}
    \DataTypeTok{names.arg =} \KeywordTok{c}\NormalTok{(}\StringTok{"No"}\NormalTok{, }\StringTok{"Sí"}\NormalTok{),}
    \DataTypeTok{ylab =} \StringTok{"Número de pacientes"}
\NormalTok{)}
\end{Highlighting}
\end{Shaded}

\includegraphics{examen_files/figure-latex/unnamed-chunk-14-4.pdf}

\begin{Shaded}
\begin{Highlighting}[]
\KeywordTok{barplot}\NormalTok{(}
    \KeywordTok{table}\NormalTok{(new\_df}\OperatorTok{$}\NormalTok{Diabetes),}
    \DataTypeTok{main =} \StringTok{"Diabetes"}\NormalTok{,}
    \DataTypeTok{names.arg =} \KeywordTok{c}\NormalTok{(}\StringTok{"No"}\NormalTok{, }\StringTok{"Sí"}\NormalTok{),}
    \DataTypeTok{ylab =} \StringTok{"Número de pacientes"}
\NormalTok{)}
\end{Highlighting}
\end{Shaded}

\includegraphics{examen_files/figure-latex/unnamed-chunk-14-5.pdf}

\begin{Shaded}
\begin{Highlighting}[]
\KeywordTok{barplot}\NormalTok{(}
    \KeywordTok{table}\NormalTok{(new\_df}\OperatorTok{$}\NormalTok{Use\_of\_stimulant\_drugs),}
    \DataTypeTok{main =} \StringTok{"Consumo de drogas estimulantes"}\NormalTok{,}
    \DataTypeTok{names.arg =} \KeywordTok{c}\NormalTok{(}\StringTok{"No"}\NormalTok{, }\StringTok{"Sí"}\NormalTok{),}
    \DataTypeTok{ylab =} \StringTok{"Número de pacientes"}
\NormalTok{)}
\end{Highlighting}
\end{Shaded}

\includegraphics{examen_files/figure-latex/unnamed-chunk-14-6.pdf}

\begin{Shaded}
\begin{Highlighting}[]
\KeywordTok{barplot}\NormalTok{(}
    \KeywordTok{table}\NormalTok{(new\_df}\OperatorTok{$}\NormalTok{Family\_history),}
    \DataTypeTok{main =} \StringTok{"Historial familiar de ataque cardíaco"}\NormalTok{,}
    \DataTypeTok{names.arg =} \KeywordTok{c}\NormalTok{(}\StringTok{"No"}\NormalTok{, }\StringTok{"Sí"}\NormalTok{),}
    \DataTypeTok{ylab =} \StringTok{"Número de pacientes"}
\NormalTok{)}
\end{Highlighting}
\end{Shaded}

\includegraphics{examen_files/figure-latex/unnamed-chunk-14-7.pdf}

\begin{Shaded}
\begin{Highlighting}[]
\KeywordTok{barplot}\NormalTok{(}
    \KeywordTok{table}\NormalTok{(new\_df}\OperatorTok{$}\NormalTok{UnderRisk),}
    \DataTypeTok{main =} \StringTok{"Riesgo de ataque cardíaco"}\NormalTok{,}
    \DataTypeTok{names.arg =} \KeywordTok{c}\NormalTok{(}\StringTok{"No"}\NormalTok{, }\StringTok{"Sí"}\NormalTok{),}
    \DataTypeTok{ylab =} \StringTok{"Número de pacientes"}
\NormalTok{)}
\end{Highlighting}
\end{Shaded}

\includegraphics{examen_files/figure-latex/unnamed-chunk-14-8.pdf}

\hypertarget{anuxe1lisis}{%
\subsection{Análisis}\label{anuxe1lisis}}

Ahora se hará un filtrado para dejar sólo las personas que están bajo
riesgo de un infarto y se mostrarán distintos gráficos

\begin{Shaded}
\begin{Highlighting}[]
\KeywordTok{library}\NormalTok{(}\StringTok{"tidyverse"}\NormalTok{)}
\KeywordTok{library}\NormalTok{(}\StringTok{"gridExtra"}\NormalTok{)}
\NormalTok{filtered \textless{}{-}}\StringTok{ }\KeywordTok{filter}\NormalTok{(new\_df, UnderRisk }\OperatorTok{==}\StringTok{ }\OtherTok{TRUE}\NormalTok{)}


\NormalTok{bar\_age \textless{}{-}}\StringTok{ }\KeywordTok{ggplot}\NormalTok{(new\_df, }\KeywordTok{aes}\NormalTok{(}\DataTypeTok{x =}\NormalTok{ Age, }\DataTypeTok{fill =}\NormalTok{ UnderRisk)) }\OperatorTok{+}
\StringTok{    }\KeywordTok{geom\_bar}\NormalTok{(}\DataTypeTok{position =} \KeywordTok{position\_dodge}\NormalTok{()) }\OperatorTok{+}
\StringTok{    }\KeywordTok{theme}\NormalTok{(}\DataTypeTok{legend.position =} \StringTok{"right"}\NormalTok{)}
\NormalTok{bar\_gender \textless{}{-}}\StringTok{ }\KeywordTok{ggplot}\NormalTok{(new\_df, }\KeywordTok{aes}\NormalTok{(}\DataTypeTok{x =}\NormalTok{ Gender, }\DataTypeTok{fill =}\NormalTok{ UnderRisk)) }\OperatorTok{+}
\StringTok{    }\KeywordTok{geom\_bar}\NormalTok{(}\DataTypeTok{position =} \KeywordTok{position\_dodge}\NormalTok{()) }\OperatorTok{+}
\StringTok{    }\KeywordTok{theme}\NormalTok{(}\DataTypeTok{legend.position =} \StringTok{"none"}\NormalTok{)}
\NormalTok{bar\_smoker \textless{}{-}}\StringTok{ }\KeywordTok{ggplot}\NormalTok{(new\_df, }\KeywordTok{aes}\NormalTok{(}\DataTypeTok{x =}\NormalTok{ Chain\_smoker, }\DataTypeTok{fill =}\NormalTok{ UnderRisk)) }\OperatorTok{+}
\StringTok{    }\KeywordTok{geom\_bar}\NormalTok{(}\DataTypeTok{position =} \KeywordTok{position\_dodge}\NormalTok{()) }\OperatorTok{+}
\StringTok{    }\KeywordTok{theme}\NormalTok{(}\DataTypeTok{legend.position =} \StringTok{"none"}\NormalTok{)}
\NormalTok{bar\_obese \textless{}{-}}\StringTok{ }\KeywordTok{ggplot}\NormalTok{(new\_df, }\KeywordTok{aes}\NormalTok{(}\DataTypeTok{x =}\NormalTok{ Obese, }\DataTypeTok{fill =}\NormalTok{ UnderRisk)) }\OperatorTok{+}
\StringTok{    }\KeywordTok{geom\_bar}\NormalTok{(}\DataTypeTok{position =} \KeywordTok{position\_dodge}\NormalTok{()) }\OperatorTok{+}
\StringTok{    }\KeywordTok{theme}\NormalTok{(}\DataTypeTok{legend.position =} \StringTok{"none"}\NormalTok{)}
\NormalTok{bar\_diabetes \textless{}{-}}\StringTok{ }\KeywordTok{ggplot}\NormalTok{(new\_df, }\KeywordTok{aes}\NormalTok{(}\DataTypeTok{x =}\NormalTok{ Diabetes, }\DataTypeTok{fill =}\NormalTok{ UnderRisk)) }\OperatorTok{+}
\StringTok{    }\KeywordTok{geom\_bar}\NormalTok{(}\DataTypeTok{position =} \KeywordTok{position\_dodge}\NormalTok{()) }\OperatorTok{+}
\StringTok{    }\KeywordTok{theme}\NormalTok{(}\DataTypeTok{legend.position =} \StringTok{"none"}\NormalTok{)}
\NormalTok{bar\_drugs \textless{}{-}}\StringTok{ }\KeywordTok{ggplot}\NormalTok{(new\_df, }\KeywordTok{aes}\NormalTok{(}\DataTypeTok{x =}\NormalTok{ Use\_of\_stimulant\_drugs, }\DataTypeTok{fill =}\NormalTok{ UnderRisk)) }\OperatorTok{+}
\StringTok{    }\KeywordTok{geom\_bar}\NormalTok{(}\DataTypeTok{position =} \KeywordTok{position\_dodge}\NormalTok{()) }\OperatorTok{+}
\StringTok{    }\KeywordTok{theme}\NormalTok{(}\DataTypeTok{legend.position =} \StringTok{"none"}\NormalTok{)}
\NormalTok{bar\_history \textless{}{-}}\StringTok{ }\KeywordTok{ggplot}\NormalTok{(new\_df, }\KeywordTok{aes}\NormalTok{(}\DataTypeTok{x =}\NormalTok{ Family\_history, }\DataTypeTok{fill =}\NormalTok{ UnderRisk)) }\OperatorTok{+}
\StringTok{    }\KeywordTok{geom\_bar}\NormalTok{(}\DataTypeTok{position =} \KeywordTok{position\_dodge}\NormalTok{()) }\OperatorTok{+}
\StringTok{    }\KeywordTok{theme}\NormalTok{(}\DataTypeTok{legend.position =} \StringTok{"none"}\NormalTok{)}

\NormalTok{bar\_age}
\end{Highlighting}
\end{Shaded}

\includegraphics{examen_files/figure-latex/unnamed-chunk-15-1.pdf}

\begin{Shaded}
\begin{Highlighting}[]
\KeywordTok{grid.arrange}\NormalTok{(}
\NormalTok{    bar\_gender,}
\NormalTok{    bar\_smoker,}
\NormalTok{    bar\_obese,}
\NormalTok{    bar\_diabetes,}
\NormalTok{    bar\_drugs,}
\NormalTok{    bar\_history)}
\end{Highlighting}
\end{Shaded}

\includegraphics{examen_files/figure-latex/unnamed-chunk-15-2.pdf}

De los gráficos se puede concluir, grosso modo, que dentro de las
personas que presentan riesgos de sufrir paros cardíacos lso factores
que más influyen son obesidad y la presencia de antecedentes familiares
de ataques cardíacos.

\hypertarget{probabilidades}{%
\subsection{Probabilidades}\label{probabilidades}}

Fallé :c

\end{document}
